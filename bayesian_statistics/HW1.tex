% Options for packages loaded elsewhere
\PassOptionsToPackage{unicode}{hyperref}
\PassOptionsToPackage{hyphens}{url}
%
\documentclass[
]{article}
\usepackage{lmodern}
\usepackage{amssymb,amsmath}
\usepackage{ifxetex,ifluatex}
\ifnum 0\ifxetex 1\fi\ifluatex 1\fi=0 % if pdftex
  \usepackage[T1]{fontenc}
  \usepackage[utf8]{inputenc}
  \usepackage{textcomp} % provide euro and other symbols
\else % if luatex or xetex
  \usepackage{unicode-math}
  \defaultfontfeatures{Scale=MatchLowercase}
  \defaultfontfeatures[\rmfamily]{Ligatures=TeX,Scale=1}
\fi
% Use upquote if available, for straight quotes in verbatim environments
\IfFileExists{upquote.sty}{\usepackage{upquote}}{}
\IfFileExists{microtype.sty}{% use microtype if available
  \usepackage[]{microtype}
  \UseMicrotypeSet[protrusion]{basicmath} % disable protrusion for tt fonts
}{}
\makeatletter
\@ifundefined{KOMAClassName}{% if non-KOMA class
  \IfFileExists{parskip.sty}{%
    \usepackage{parskip}
  }{% else
    \setlength{\parindent}{0pt}
    \setlength{\parskip}{6pt plus 2pt minus 1pt}}
}{% if KOMA class
  \KOMAoptions{parskip=half}}
\makeatother
\usepackage{xcolor}
\IfFileExists{xurl.sty}{\usepackage{xurl}}{} % add URL line breaks if available
\IfFileExists{bookmark.sty}{\usepackage{bookmark}}{\usepackage{hyperref}}
\hypersetup{
  pdftitle={STAT641 - Homework 1},
  pdfauthor={Marcel Gietzmann-Sanders},
  hidelinks,
  pdfcreator={LaTeX via pandoc}}
\urlstyle{same} % disable monospaced font for URLs
\usepackage[margin=1in]{geometry}
\usepackage{color}
\usepackage{fancyvrb}
\newcommand{\VerbBar}{|}
\newcommand{\VERB}{\Verb[commandchars=\\\{\}]}
\DefineVerbatimEnvironment{Highlighting}{Verbatim}{commandchars=\\\{\}}
% Add ',fontsize=\small' for more characters per line
\usepackage{framed}
\definecolor{shadecolor}{RGB}{248,248,248}
\newenvironment{Shaded}{\begin{snugshade}}{\end{snugshade}}
\newcommand{\AlertTok}[1]{\textcolor[rgb]{0.94,0.16,0.16}{#1}}
\newcommand{\AnnotationTok}[1]{\textcolor[rgb]{0.56,0.35,0.01}{\textbf{\textit{#1}}}}
\newcommand{\AttributeTok}[1]{\textcolor[rgb]{0.77,0.63,0.00}{#1}}
\newcommand{\BaseNTok}[1]{\textcolor[rgb]{0.00,0.00,0.81}{#1}}
\newcommand{\BuiltInTok}[1]{#1}
\newcommand{\CharTok}[1]{\textcolor[rgb]{0.31,0.60,0.02}{#1}}
\newcommand{\CommentTok}[1]{\textcolor[rgb]{0.56,0.35,0.01}{\textit{#1}}}
\newcommand{\CommentVarTok}[1]{\textcolor[rgb]{0.56,0.35,0.01}{\textbf{\textit{#1}}}}
\newcommand{\ConstantTok}[1]{\textcolor[rgb]{0.00,0.00,0.00}{#1}}
\newcommand{\ControlFlowTok}[1]{\textcolor[rgb]{0.13,0.29,0.53}{\textbf{#1}}}
\newcommand{\DataTypeTok}[1]{\textcolor[rgb]{0.13,0.29,0.53}{#1}}
\newcommand{\DecValTok}[1]{\textcolor[rgb]{0.00,0.00,0.81}{#1}}
\newcommand{\DocumentationTok}[1]{\textcolor[rgb]{0.56,0.35,0.01}{\textbf{\textit{#1}}}}
\newcommand{\ErrorTok}[1]{\textcolor[rgb]{0.64,0.00,0.00}{\textbf{#1}}}
\newcommand{\ExtensionTok}[1]{#1}
\newcommand{\FloatTok}[1]{\textcolor[rgb]{0.00,0.00,0.81}{#1}}
\newcommand{\FunctionTok}[1]{\textcolor[rgb]{0.00,0.00,0.00}{#1}}
\newcommand{\ImportTok}[1]{#1}
\newcommand{\InformationTok}[1]{\textcolor[rgb]{0.56,0.35,0.01}{\textbf{\textit{#1}}}}
\newcommand{\KeywordTok}[1]{\textcolor[rgb]{0.13,0.29,0.53}{\textbf{#1}}}
\newcommand{\NormalTok}[1]{#1}
\newcommand{\OperatorTok}[1]{\textcolor[rgb]{0.81,0.36,0.00}{\textbf{#1}}}
\newcommand{\OtherTok}[1]{\textcolor[rgb]{0.56,0.35,0.01}{#1}}
\newcommand{\PreprocessorTok}[1]{\textcolor[rgb]{0.56,0.35,0.01}{\textit{#1}}}
\newcommand{\RegionMarkerTok}[1]{#1}
\newcommand{\SpecialCharTok}[1]{\textcolor[rgb]{0.00,0.00,0.00}{#1}}
\newcommand{\SpecialStringTok}[1]{\textcolor[rgb]{0.31,0.60,0.02}{#1}}
\newcommand{\StringTok}[1]{\textcolor[rgb]{0.31,0.60,0.02}{#1}}
\newcommand{\VariableTok}[1]{\textcolor[rgb]{0.00,0.00,0.00}{#1}}
\newcommand{\VerbatimStringTok}[1]{\textcolor[rgb]{0.31,0.60,0.02}{#1}}
\newcommand{\WarningTok}[1]{\textcolor[rgb]{0.56,0.35,0.01}{\textbf{\textit{#1}}}}
\usepackage{graphicx}
\makeatletter
\def\maxwidth{\ifdim\Gin@nat@width>\linewidth\linewidth\else\Gin@nat@width\fi}
\def\maxheight{\ifdim\Gin@nat@height>\textheight\textheight\else\Gin@nat@height\fi}
\makeatother
% Scale images if necessary, so that they will not overflow the page
% margins by default, and it is still possible to overwrite the defaults
% using explicit options in \includegraphics[width, height, ...]{}
\setkeys{Gin}{width=\maxwidth,height=\maxheight,keepaspectratio}
% Set default figure placement to htbp
\makeatletter
\def\fps@figure{htbp}
\makeatother
\setlength{\emergencystretch}{3em} % prevent overfull lines
\providecommand{\tightlist}{%
  \setlength{\itemsep}{0pt}\setlength{\parskip}{0pt}}
\setcounter{secnumdepth}{-\maxdimen} % remove section numbering

\title{STAT641 - Homework 1}
\author{Marcel Gietzmann-Sanders}
\date{}

\begin{document}
\maketitle

\hypertarget{general-questions}{%
\section{1. General Questions}\label{general-questions}}

\begin{enumerate}
\def\labelenumi{\alph{enumi}.}
\tightlist
\item
  Marcel Gietzmann-Sanders - 31308957
\item
\item
  No
\item
  Just wanting a broad introduction to the subject.
\item
  I'm studying fisheries and frequentists stats is everywhere. I'd like
  to get a better handle on the Bayesian side of things.
\item
  Yes
\item
  This will be the first time.
\item
  Yes
\item
  I've been a software engineer working primarily in Python for the past
  7 years. All of my classes here have used R. So at this point I
  probably have over 10,000 hours of Python and several hundred hours of
  R experience.
\item
  I'm on a macbook but I typically run a Linux VM for my work.
\item
  No
\end{enumerate}

\hypertarget{opening-problems}{%
\section{2. Opening Problems}\label{opening-problems}}

\hypertarget{a-use-r-to-simulate-the-following-state-the-values-it-generates.}{%
\subsubsection{(a) Use R to simulate the following; state the values it
generates.}\label{a-use-r-to-simulate-the-following-state-the-values-it-generates.}}

\begin{Shaded}
\begin{Highlighting}[]
\KeywordTok{rbinom}\NormalTok{(}\DecValTok{5}\NormalTok{, }\DataTypeTok{size=}\DecValTok{12}\NormalTok{, }\DataTypeTok{prob=}\FloatTok{0.4}\NormalTok{)}
\end{Highlighting}
\end{Shaded}

\begin{verbatim}
## [1] 4 3 4 1 5
\end{verbatim}

\(n\) is the number of trials per experiment and \(p\) is the
probability of a success (or equivalently, of a 1) per trial.

\begin{Shaded}
\begin{Highlighting}[]
\KeywordTok{rnorm}\NormalTok{(}\DecValTok{5}\NormalTok{, }\DataTypeTok{mean=}\FloatTok{0.0}\NormalTok{, }\DataTypeTok{sd=}\FloatTok{0.75}\NormalTok{)}
\end{Highlighting}
\end{Shaded}

\begin{verbatim}
## [1]  0.6877932  0.8038130  0.6556525  0.6992979 -0.4075211
\end{verbatim}

\hypertarget{b-confidence-intervals}{%
\subsubsection{(b) Confidence Intervals}\label{b-confidence-intervals}}

\begin{quote}
I purchase a random sample of n = 49 avocados at Fred Meyer. The sample
mean weight is 4.3 ounces with a standard deviation of 0.36 ounces.
Calculate a 95\% confidence interval for the mean weight of all such
avocados, and interpret your confidence interval. (Be sure to state the
formula you're using, and fill in the various values in the formula.)
What, exactly, is it that happens with 95\% probability? What does the
``95\%'' refer to when constructing a 95\% confidence interval.
\end{quote}

The formula for a 95\% confidence interval is:

\[\bar{\mu} \pm Z_{0.975}\frac{s}{\sqrt{n}}\]

where \(\bar{\mu}\) is the sample mean, \(s\) is the sample standard
deviation, and \(Z_{0.975}\) is the z-score at which the cumulative
probability of a normal distribution is 97.5\%. Because this is two
tailed, choosing the 97.5\% ensures that we are covering 95\% of the
distribution.

In our case this becomes:

\[4.3\pm 1.96 \frac{0.36}{\sqrt{49}}=4.3\pm 0.1008\]

So this is saying (informally) that we are 95\% confident that the true
mean weight of avocados at Fred Meyer lies somewhere between 4.1992 and
4.4008 ounces. Formally if we were to repeat this experiment over and
over 95\% of the time the true mean weight would lie within the
confidence interval derived in that experiment.

\hypertarget{c-hypothesis-testing}{%
\subsubsection{(c) Hypothesis Testing}\label{c-hypothesis-testing}}

\begin{quote}
I conducted a poll consisting of a random sample of n = 1000 sane
individuals. 790 of those polled answered ``Yes'' to the question, ``Is
the world going insane?'' Test the hypotheses, H0 : p = 0.75 versus Ha :
p \textgreater{} 0.75, where p is the proportion of all sane individuals
who believe the world is going insane. (It's okay to conduct the test by
hand or to use the R function prop.test.) At level α = 0.05, do we
reject H0 or fail to reject H0?

It turns out that the p-value for this test is 0.00174. At level α =
0.01, should we reject H0? Why or why not?
\end{quote}

\begin{Shaded}
\begin{Highlighting}[]
\KeywordTok{prop.test}\NormalTok{(}\DecValTok{790}\NormalTok{, }\DecValTok{1000}\NormalTok{, }\DataTypeTok{p=}\FloatTok{0.75}\NormalTok{, }\DataTypeTok{alternative=}\StringTok{"greater"}\NormalTok{)}
\end{Highlighting}
\end{Shaded}

\begin{verbatim}
## 
##  1-sample proportions test with continuity correction
## 
## data:  790 out of 1000, null probability 0.75
## X-squared = 8.3213, df = 1, p-value = 0.001959
## alternative hypothesis: true p is greater than 0.75
## 95 percent confidence interval:
##  0.7675285 1.0000000
## sample estimates:
##    p 
## 0.79
\end{verbatim}

In this case the p-value is 0.001959 which is far below our
\(\alpha=0.05\) threshold. Therefore we can go ahead and reject the null
hypothesis.

With either 0.00174 or our 0.001959 we should reject the null hypothesis
given an \(\alpha=0.01\) because we are still below the threshold.

\end{document}
